% ŠABLONA PRO PSANÍ ZÁVĚREČNÉ STUDIJNÍ PRÁCE
%%%%%%%%%%%%%%%%%%%%%%%%%%%%%%%%%%%%%%%%%%%%
% Autor: Jakub Dokulil (kubadokulil99@gmail.com)
% Tato šablona byla vytvořena tak, aby pomocí ní mohli v systému LaTeX soutěžící sázet své práce a zároveň odpovídala požadavkům na formátování vyplývajícím z wordové šablony umístěné na webu soc.cz.
%
\documentclass[12pt, a4paper,
%oneside,      %% -- odkomentujte, pokud chcete svou práci mít pouze jednostrannou, mezera pro hřbet pak automaticky bude pouze na levé straně
twoside,        %% -- pro oboustranné práce, mezera pro hřbet následně střídá strany.
openright
]{report}

%% Nutné balíčky a nastavení
%%%%%%%%%%%%%%%%%%%%%%%%%%%%

%% Proměnné
\newcommand\obor{INFORMAČNÍ TECHNOLOGIE} %% -- napiš číslo a název tvého oboru
\newcommand\kodOboru{18-20-M/01} %% -- napiš číslo a název tvého oboru
\newcommand\zamereni{se zaměřením na počítačové sítě a programování} %% -- napiš číslo a název tvého oboru
\newcommand\skola{Střední škola průmyslová a umělecká, Opava} %% vyplň název školy
\newcommand\trida{IT4} %% vyplň jméno svého konzultanta
\newcommand\jmenoAutora{Jan Stránský}  %% vyplň své jméno
\newcommand\skolniRok{2024/25} %% vyplň rok
\newcommand\datumOdevzdani{1. 1. 2025} %% vyplň rok
\newcommand\nazevPrace{CTF systém v Kubernetes} %% vyplň název své práce

\title{\nazevPrace} %% -- Název tvé práce
\author{\jmenoAutora} %% -- tvé jméno
\date{\datumOdevzdani} %% -- rok, kdy píšeš SOČku

\usepackage[top=2.5cm, bottom=2.5cm, left=3.5cm, right=1.5cm]{geometry} %% nastaví okraje, left -- vnitřní okraj, right -- vnější okraj

\usepackage[czech]{babel} %% balík babel pro sazbu v češtině
\usepackage[utf8]{inputenc} %% balíky pro kódování textu
\usepackage[T1]{fontenc}
\usepackage{cmap} %% balíček zajišťující, že vytvořené PDF bude prohledávatelné a kopírovatelné

\usepackage{graphicx} %% balík pro vkládání obrázků

\usepackage{subcaption} %% balíček pro vkládání podobrázků

\usepackage{hyperref} %% balíček, který v PDF vytváří odkazy

\linespread{1.25} %% řádkování
\setlength{\parskip}{0.5em} %% odsazení mezi odstavci


\usepackage[pagestyles]{titlesec} %% balíček pro úpravu stylu kapitol a sekcí
\titleformat{\chapter}[block]{\scshape\bfseries\LARGE}{\thechapter}{10pt}{\vspace{0pt}}[\vspace{-22pt}]
\titleformat{\section}[block]{\scshape\bfseries\Large}{\thesection}{10pt}{\vspace{0pt}}
\titleformat{\subsection}[block]{\bfseries\large}{\thesubsection}{10pt}{\vspace{0pt}}


\usepackage{tocloft} % Balíček umožní přizpůsobit vzhled tabulky obsahu
\setlength{\cftbeforechapskip}{0pt}  % Menší rozestup pro kapitoly
\setlength{\cftbeforesecskip}{0pt}   % Menší rozestup pro sekce

\setcounter{secnumdepth}{2}
\setcounter{tocdepth}{1}
\usepackage{fancyhdr}
\pagestyle{fancy}
\renewcommand{\headrulewidth}{0.025pt}

\usepackage{booktabs}

\usepackage{url}

%% Balíčky co se můžou hodit :) 
%%%%%%%%%%%%%%%%%%%%%%%%%%%%%%%

\usepackage{pdfpages} %% Balíček umožňující vkládat stránky z PDF souborů, 

\usepackage{upgreek} %% Balíček pro sazbu stojatých řeckých písmen, třeba u jednotky mikrometr. Například stojaté mí: \upmu, stojaté pí: \uppi

\usepackage{amsmath}    %% Balíčky amsmath a amsfonts 
\usepackage{amsfonts}   %% pro sazbu matematických symbolů
\usepackage{esint}     %% pro sazbu různých integrálů (např \oiint)
\usepackage{mathrsfs}
\usepackage{helvet} % Helvet font
\usepackage{mathptmx} % Times New Roman
\usepackage{Oswald} % Oswald font


%% makra pro sazbu matematiky
\newcommand{\dif}{\mathrm{d}} %% makro pro sazbu diferenciálu, místo toho
%% abych musel psát '\mathrm{d}' mi stačí napsat '\dif' což je mnohem 
%% kratší a mohu si tak usnadnit práci

\usepackage{listings}
\usepackage{xcolor}

\renewcommand{\lstlistingname}{Kód}% Listing -> Algorithm
\renewcommand{\lstlistlistingname}{Seznam programových kódů}% List of Listings -> List of Algorithms

%% Definice 
\lstdefinelanguage{JavaScript}{
	morekeywords=[1]{break, continue, delete, else, for, function, if, in,
		new, return, this, typeof, var, void, while, with},
	% Literals, primitive types, and reference types.
	morekeywords=[2]{false, null, true, boolean, number, undefined,
		Array, Boolean, Date, Math, Number, String, Object},
	% Built-ins.
	morekeywords=[3]{eval, parseInt, parseFloat, escape, unescape},
	sensitive,
	morecomment=[s]{/*}{*/},
	morecomment=[l]//,
	morecomment=[s]{/**}{*/}, % JavaDoc style comments
	morestring=[b]',
	morestring=[b]"
}[keywords, comments, strings]


\lstdefinelanguage[ECMAScript2015]{JavaScript}[]{JavaScript}{
	morekeywords=[1]{await, async, case, catch, class, const, default, do,
		enum, export, extends, finally, from, implements, import, instanceof,
		let, static, super, switch, throw, try},
	morestring=[b]` % Interpolation strings.
}

\lstalias[]{ES6}[ECMAScript2015]{JavaScript}

% Nastavení barev
% Requires package: color.
\definecolor{mediumgray}{rgb}{0.3, 0.4, 0.4}
\definecolor{mediumblue}{rgb}{0.0, 0.0, 0.8}
\definecolor{forestgreen}{rgb}{0.13, 0.55, 0.13}
\definecolor{darkviolet}{rgb}{0.58, 0.0, 0.83}
\definecolor{royalblue}{rgb}{0.25, 0.41, 0.88}
\definecolor{crimson}{rgb}{0.86, 0.8, 0.24}

% Nastavení pro Python
\lstdefinestyle{Python}{
	language=Python,
	backgroundcolor=\color{white},
	basicstyle=\ttfamily,
	breakatwhitespace=false,
	breaklines=false,
	captionpos=b,
	columns=fullflexible,
	commentstyle=\color{mediumgray}\upshape,
	emph={},
	emphstyle=\color{crimson},
	extendedchars=true,  % requires inputenc
	fontadjust=true,
	frame=single,
	identifierstyle=\color{black},
	keepspaces=true,
	keywordstyle=\color{mediumblue},
	keywordstyle={[2]\color{darkviolet}},
	keywordstyle={[3]\color{royalblue}},
	literate=%
	{á}{{\'a}}1 {č}{{\v{c}}}1 {ď}{{\v{d}}}1 {é}{{\'e}}1 {ě}{{\v{e}}}1
	{í}{{\'i}}1 {ň}{{\v{n}}}1 {ó}{{\'o}}1 {ř}{{\v{r}}}1 {š}{{\v{s}}}1
	{ť}{{\v{t}}}1 {ú}{{\'u}}1 {ů}{{\r{u}}}1 {ý}{{\'y}}1 {ž}{{\v{z}}}1,		
	numbers=left,
	numbersep=5pt,
	numberstyle=\tiny\color{black},
	rulecolor=\color{black},
	showlines=true,
	showspaces=false,
	showstringspaces=false,
	showtabs=false,
	stringstyle=\color{forestgreen},
	tabsize=2,
	title=\lstname,
	upquote=true  % requires textcomp	
}


\lstdefinestyle{JSES6Base}{
	backgroundcolor=\color{white},
	basicstyle=\ttfamily,
	breakatwhitespace=false,
	breaklines=false,
	captionpos=b,
	columns=fullflexible,
	commentstyle=\color{mediumgray}\upshape,
	emph={},
	emphstyle=\color{crimson},
	extendedchars=true,  % requires inputenc
	fontadjust=true,
	frame=single,
	identifierstyle=\color{black},
	keepspaces=true,
	keywordstyle=\color{mediumblue},
	keywordstyle={[2]\color{darkviolet}},
	keywordstyle={[3]\color{royalblue}},
 literate=%
{á}{{\'a}}1 {č}{{\v{c}}}1 {ď}{{\v{d}}}1 {é}{{\'e}}1 {ě}{{\v{e}}}1
{í}{{\'i}}1 {ň}{{\v{n}}}1 {ó}{{\'o}}1 {ř}{{\v{r}}}1 {š}{{\v{s}}}1
{ť}{{\v{t}}}1 {ú}{{\'u}}1 {ů}{{\r{u}}}1 {ý}{{\'y}}1 {ž}{{\v{z}}}1,		
	numbers=left,
	numbersep=5pt,
	numberstyle=\tiny\color{black},
	rulecolor=\color{black},
	showlines=true,
	showspaces=false,
	showstringspaces=false,
	showtabs=false,
	stringstyle=\color{forestgreen},
	tabsize=2,
	title=\lstname,
	upquote=true  % requires textcomp
}

\lstdefinestyle{JavaScript}{
	language=JavaScript,
	style=JSES6Base,
}
\lstdefinestyle{ES6}{
	language=ES6,
	style=JSES6Base
}


%% Bordel pro práci - můžeš smáznout :) 
%%%%%%%%%%%%%%%%%%%

\usepackage{lipsum} %% balíček který píše lipsum (nesmyslný text, který se používá pro kontrolu typografie)

%% Začátek dokumentu
%%%%%%%%%%%%%%%%%%%%
\begin{document}
	
	\pagestyle{empty}
	\pagenumbering{Roman}
	
	\cleardoublepage

%% Titulní stránka s informacemi
%%%%%%%%%%%%%%%%%%%%%%%%%%%%%%%%%%%%%%%%
	
	{\fontfamily{phv}\selectfont
		%% Logo školy
		\begin{figure}[h]
			\centering
			\includegraphics[width=0.6\linewidth]{image/logo-skoly.png} 
		\end{figure}
		
		
		%% Hlavička práce a její název (viz proměnná \nazev prace)
		%% \sffamily %%% bezpatkové písmo - sans serif
		{\bfseries %%% písmo na stránce je tučně
			\begin{center}
				\vspace{0.025 \textheight}
				\LARGE{ZÁVĚREČNÁ STUDIJNÍ PRÁCE}\\
				\large{dokumentace}\\
				\vspace{0.075 \textheight}
				\LARGE {\nazevPrace}\\
			\end{center}  
		}%%%
		
		\begin{figure}[h]
			\centering
			\includegraphics[width=0.8\linewidth]{image/programovani-02.jpg} 
		\end{figure}
		
		\vspace{0.02 \textheight}
		\begin{table}[h!]
			\begin{tabular}{ll}
				\textbf{Autor:} & \jmenoAutora\\ 
				\textbf{Obor:} & \kodOboru { } \obor\\
				\textbf{} & \zamereni\\
				\textbf{Třída:} & \trida\\
				\textbf{Školní rok:} & \skolniRok\\
			\end{tabular}
			
		\end{table}		
	}
	
\cleardoublepage %% Zalomení dvojstránky
	
%% Stránka obsahující poděkování a prohlášení
%%%%%%%%%%%%%%%%%%%%%%%%%%%%%%%%%%%%%%%%%%%%%%%%%%%%%%%%

%% Poděkování - nepovinné
%%%%%%%%%%%%%%%%%%%%%%%%%%%%
	
	\noindent{\large{\bfseries{Poděkování}\\}}
	\noindent Rád bych poděkoval pánům učitelům Ing. Petru Grussmannovi a Mgr. Marku Lučnému za jejich pomoc s projektem, jelikož mi poskytovali cenné rady a připomínky.

	
	\vspace*{0.7\textheight} %% Vertikální mezeru je možné upravit

%% Prohlášení - povinné
%%%%%%%%%%%%%%%%%%%%%%%%%%%%
	\noindent{\large{\bfseries{Prohlášení}\\}}  %% uprav si koncovky podle toho na jaký rod se cítíš, vypadá to pak lépe :) 
	\noindent{Prohlašuji, že jsem závěrečnou práci vypracoval samostatně a uvedl veškeré použité 
		informační zdroje.\\}
	\noindent{Souhlasím, aby tato studijní práce byla použita k výukovým a prezentačním účelům na Střední průmyslové a umělecké škole v Opavě, Praskova 399/8.}
	\vfill
	\noindent{V Opavě \datumOdevzdani\\}
	\noindent
	\begin{minipage}{\linewidth}
		\hspace{9.5cm} 
		\begin{tabular}{@{}p{6cm}@{}}
			\dotfill \\
			Podpis autora
		\end{tabular}
	\end{minipage}
	
	\cleardoublepage %% Zalomení dvojstránky

%% Stránka obsahující abstrakt (anotaci)
%%%%%%%%%%%%%%%%%%%%%%%%%%%%%%%%%%%%%%%%%%%%%%%%%%%%%%%%	

%% Abstrakt v češtině
%%%%%%%%%%%%%%%%%%%%%%%%%%%%
	\noindent{\Large{\bfseries{Abstrakt}\\}}
	\noindent Výsledkem tohoto projektu je funkční systém pro spouštění a vytváření úloh CTF typu v systému Kubernetes běžícím na školní síti s dostatečnou mírou zabezpečení. Aplikace zahrnuje registraci a přihlašování uživatelů, zapínání nových úloh a následně jejich vypínání. Hlavní částí tohoto projektu je komunikace se systémem Kubernetes, který se využívá ve vysoce škálovaných produkčních prostředích. Uživatel s aplikací může komunikovat skrz poskytnuté webové prostředí, ale může komunikovat i přímo s poskytnutou API. Dále si tento projekt klade za cíl umožnit studentům se lépe seznámit s určitými možnostmi v oblasti IT formou hry (CTF) jako to dělají služby jako např. TryHackMe nebo HackTheBox.
	
	\vspace{18pt}
	
	\noindent{\large{\bfseries{Klíčová slova}}}
	
	\noindent CTF, Kubernetes, FastAPI, webová aplikace
	
	\vspace{18pt}

%% Abstrakt v angličtině
%%%%%%%%%%%%%%%%%%%%%%%%%%%%	
	\noindent{\Large{\bfseries{Abstract}}}
	
	\noindent 

	
	\vspace{18pt}
	
	\noindent{\large{\bfseries{Keywords}}}
	
	\noindent Template, \LaTeX, High school proffessional activity, \dots 
	
	\clearpage %% Zalomení stránky

%% Stránka s generovaným obsahem
%%%%%%%%%%%%%%%%%%%%%%%%%%%%%%%%%%%%%%%	
	
	\tableofcontents %% Vygeneruje tabulku s obsahem

	\pagenumbering{arabic} %% Nastavení způsobu číslování stránek (alternativy roman | Roman)
	\setcounter{page}{1} %% Nastavení počitadla stránek

%% Stránka s úvodem - povinná část
%%%%%%%%%%%%%%%%%%%%%%%%%%%%%%%%%%%%%%%		
	\chapter*{Úvod}
%Tento příkaz vytvoří novou kapitolu s názvem "Úvod" ve vašem dokumentu.
%Hvězdička * u příkazu \chapter* znamená, že tato kapitola nebude mít číslo. Ve výsledném dokumentu se tedy objeví jako "Úvod" bez předcházejícího čísla kapitoly, které se obvykle zobrazuje u číslovaných kapitol.
%Tento příkaz také znamená, že kapitola se automaticky neobjeví v obsahu, protože LaTeX standardně zahrnuje do obsahu pouze číslované kapitoly.
	\addcontentsline{toc}{chapter}{Úvod}
%Tento příkaz ručně přidává záznam do obsahu.
%První parametr toc označuje, že přidáváme záznam do Table of Contents (obsahu).
%Druhý parametr chapter specifikuje úroveň záznamu. V tomto případě říkáme, že přidávaný záznam má být považován za kapitolu.
%Třetí parametr Úvod je text, který se objeví v obsahu. V tomto případě bude v obsahu zobrazen název "Úvod".	

Mým cílem v této práci bylo sestavit škálovatelný software, který by nad prostředím Kubernetes vytvářet a spravovat kontejnery pro soutěž typu CTF (Capture The Flag). Zároveň bylo cílem, aby se tento software dal nasadit i v prostředí s nízkým oprávněním a aby ho šlo škálovat díky architektuře mikroslužeb.

Hlavní motivací bylo pochopení funkce a komunikace v rámci aplikací s formátem typu mikroslužeb místo monolitických aplikací a zlepšení svých dovedností v oblasti prostředí Kubernetes.

Zvláštní zaměření bylo na backendovou část API a na zabezpečení celého systému.

%Tipy k psaní úvodu
%Je povinný, nadpis neměňte, rozsah - max. 1 strana. 
%Tato část práce obsahuje: 
%* náhled do řešené problematiky, zdůvodnění volby problematiky, 
%* předem definované cíle práce, 
%* motivaci pro další čtení textu včetně stručného uvedení obsahu následujících kapitol 


\chapter{Backend mikroslužby}

\section{Úvod}
\label{sec:uvod}

V této kapitole se seznámíme s tím, co mikroslužby jsou a s jednotlivými mikroslužbami použitými v API částí tohoto projektu. Všechny tyto mikroslužby jsou napsány v jazyce Python s použitím 

Tyto mikroslužby jsou:
\begin{itemize}
	\item Router
	\item Auth
	\item Lister
	\item Deployer 
	\item Deleter
	\item Flag-submitter
\end{itemize}

\section{Router}
Tato mikroslužba je zodpovědná za směrování požadavků na správné mikroslužby a veškeré požadavky na API putují skrz ni, díky čemuž se dá využít globální modifikace, monitorování a logování požadavků. Kvůli tomuto účelu tato služba nepotřebuje žádné privilegované přístupy do ostatních částí systému. Jednou z částí této mikroslužby je i zajištění přesunu JWT tokenu z cookie do hlavičky požadavku, aby se dala API používat jak z webového frontendu, tak i z jiných aplikací.

Tato mikroslužba zároveň funguje jakožto filtr nevalidních typů požadavků (dále posílá pouze požadavky typu GET, POST, PUT a DELETE, ostatní jsou zahozeny s chybovou hláškou)

\section{Auth}
Tato mikroslužba je zodpovědná za registraci uživatele a vytvářením jeho záznamu v databázi PostgreSQL. 

Tato služba je jediná, která má přístup k privátnímu klíči použivaného k podepisování tokenů algoritmem RS256. Dále je také zodpovědná za ověření přihlašovacích údajů uživatele a vytvoření JWT tokenu, který se následně používá pro ověření uživatele v ostatních částech systému. Tato služba má přístup k databázi PostgreSQL.

Tato služba má tři API endpointy:
\begin{itemize}
	\item POST /register
	\item POST /login
	\item GET /health
\end{itemize}
kde první dva slouží k registraci a přihlášení uživatele a třetí slouží k zjištění stavu služby, primárně kvůli liveness a readiness HTTP checku v Kubernetes při chybě nebo při čekání na databázi.

\section{Lister}
Účel mikroslužby Lister je umožnění uživatelům získat informace o všech dostupných úlohách a jejich stavech. Dále tato služba umožňuje získat data o právě aktivních úlohách uživatele a získání detailních informací o těchto úlohách.

Tato mikroslužba potřebuje přístup k Redis a PostgreSQL databázím.

Tato služba má čtyři API endpointy:
\begin{itemize}
	\item GET /
	\item GET /running
	\item GET /running/{id}
	\item GET /health
\end{itemize}
kde první endpoint vrací veškeré dostupné úlohy a nepotřebuje žádné přihlášení, zatímco druhý a třetí endpoint vrací informace o právě běžících úkolech uživatele, tudíž vyžadují token, s tím, že třetí vrací i detailní informace o tomto úkolu.

\section{Deployer}
Tato mikroslužba zajišťuje zapínání úkolů uživatele v systému Kubernetes a zapsání informací o této běžící službě do databáze Redis, čímž zpřístupní tato data službě Lister.

Jednotlivé úkoly jsou v Kubernetes spuštěné jakožto pody v namespace daným uživatelem, což je také jeden z důvodů užívání samostatného Kubernetes clusteru (ať už opravdového nebo vcluster) pro tyto studentské stroje - ServiceAccount  spojený s tímto projektem musí mít jak práva na vytváření nových podů, tak vytváření nových namespace.

Tato služba vyžaduje přístup k Redis a PostgreSQL databázím a ke Kubernetes API.

Tato služba má dva API endpointy:
\begin{itemize}
	\item POST /
	\item GET /health
\end{itemize}
kde základní endpoint vyžaduje JSON data s \texttt{challenge\_id} klíčem. Dále tento endpoint potřebuje přístup k tokenu.

\section{Deleter}
Tato mikroslužba umožňuje vypínat (mazat) již vytvořené úkoly uživatele a to jak v Redis databázi, tak jejich instance běžící v systému Kubernetes.

Tato služba vyžaduje přístup k Redis databázi a ke Kubernetes API.

Tato služba má dva API endpointy:
\begin{itemize}
	\item DELETE /{id}
	\item GET /health
\end{itemize}
kde endpoint \texttt{/\{id\}} vyžaduje id úkolu, který uživatel chce vypnout a JWT token uživatele.

\section{Flag-submitter}
Tato mikroslužba umožňuje odevzdávat řešení jednotlivých úkolů (vlajky).

Tato služba vyžaduje přístup k PostgreSQL databázi.

Tato služba má dva API endpointy:
\begin{itemize}
	\item POST /{flag\_id}
	\item GET /health
\end{itemize}
kde endpoint \texttt{/\{flag\_id\}} vyžaduje v těle požadavku string \texttt{flag} a token uživatele.

\chapter{Frontend}
Frontend je napsaný pomocí Vite React templatu a umožňuje uživatelům interagovat s jedntlivými částmi API.

\chapter{Administrátorská Sekce}
Sekce pro správce ještě není vytvořená, ale bude umožňovat administrátorovi přidávat nové úlohy nahráváním Kubernetes manifestů a umožní správci sledovat stav uživatelské části stránky (např. počet zapnutých úkolů).
	
	\chapter*{Závěr}

	Cílem práce je webová aplikace a REST API pro práci s CTF systémem postaveným na platformě Kubernetes.

	Aplikace je zálohavaná na GitHubu na adrese https://github.com/jan1s2-maturita
	

	
	%% literatura
	%\begin{thebibliography}{99}
	%\end{thebibliography}
	
	%%% obrázky 
	%\listoffigures
	
	%%% tabulky
	%\listoftables
	
	\appendix %% začínají přílohy
	
	\titleformat{\chapter}[block]{\scshape\bfseries\LARGE}{Příloha \thechapter}{10pt}{\vspace{0pt}}[\vspace{-22pt}] %% nastavení nadpisu u příloh
	
	
	%\chapter{%Příloha A 
		%Spot diagramy a další }
	
	
\end{document}
