\documentclass[notes,12pt]{beamer}
\usetheme{Madrid}

\usepackage{pifont}
\usepackage[framemethod=TikZ]{mdframed}

% multiple optional arguments
\usepackage{xparse}

% single quote
\usepackage{textcomp}

% chemie
\usepackage{chemfig}
\usepackage{listings}
\usepackage[version=4]{mhchem}
\usepackage{bohr}
\ExplSyntaxOn
\keys_define:nn { mhchem }
 {
  arrow-min-length .code:n =
   \cs_set:Npn \__mhchem_arrow_options_minLength:n { {#1} } % default is 2em
 }
\ExplSyntaxOff
\mhchemoptions{arrow-min-length=1em}

\usepackage[font=scriptsize,skip=5pt]{caption}
\newcommand\numberthis{\addtocounter{equation}{1}\tag{\theequation}}
\setbeamertemplate{itemize item}{\ding{70}}
\setbeamertemplate{itemize subitem}{\ding{81}}

\graphicspath{ {images/} }

\setbeamertemplate{headline}{}
\setbeamertemplate{navigation symbols}{}

% Nastavení pro Python
\lstdefinestyle{Python}{
	language=Python,
	backgroundcolor=\color{white},
	basicstyle=\ttfamily,
	breakatwhitespace=false,
	breaklines=false,
	captionpos=b,
	columns=fullflexible,
	commentstyle=\color{mediumgray}\upshape,
	emph={},
	emphstyle=\color{crimson},
	extendedchars=true,  % requires inputenc
	fontadjust=true,
	frame=single,
	identifierstyle=\color{black},
	keepspaces=true,
	keywordstyle=\color{mediumblue},
	keywordstyle={[2]\color{darkviolet}},
	keywordstyle={[3]\color{royalblue}},
	literate=%
	{á}{{\'a}}1 {č}{{\v{c}}}1 {ď}{{\v{d}}}1 {é}{{\'e}}1 {ě}{{\v{e}}}1
	{í}{{\'i}}1 {ň}{{\v{n}}}1 {ó}{{\'o}}1 {ř}{{\v{r}}}1 {š}{{\v{s}}}1
	{ť}{{\v{t}}}1 {ú}{{\'u}}1 {ů}{{\r{u}}}1 {ý}{{\'y}}1 {ž}{{\v{z}}}1,		
	numbers=left,
	numbersep=5pt,
	numberstyle=\tiny\color{black},
	rulecolor=\color{black},
	showlines=true,
	showspaces=false,
	showstringspaces=false,
	showtabs=false,
	stringstyle=\color{forestgreen},
	tabsize=2,
	title=\lstname,
	upquote=true  % requires textcomp	
}
\NewDocumentCommand{\col}{O{0.5} m}{
        \begin{column}{#1\textwidth}
            #2
        \end{column}
}

\NewDocumentCommand{\tcol}{O{0.5} m}{
    \col[#1]{
            \begin{itemize}
                    #2
            \end{itemize}
        }
}

\NewDocumentCommand{\icol}{O{0.5} O{1} m m}{
    \col[#1]{
            \begin{figure}
                \centering
                \includegraphics[width=#2\textwidth,height=0.7\paperheight,keepaspectratio]{#3}
                \caption{#4}
                \label{fig:#3}
            \end{figure}
    }
}

% first argument is the associated text, second is the image name and the third is the image caption
\NewDocumentCommand{\cols}{m m m}{
    \begin{columns}
        \tcol{#1}
        \icol{#2}{#3}
    \end{columns}
}


\begin{document}
\title{CTF systém v Kubernetes}
\author{Jan Stránský}
\date{}

% Title slide
\begin{frame}[noframenumbering]
\titlepage
\end{frame}

\begin{frame}{CTF}
    \begin{columns}
        \tcol{
            \item Capture The Flag
            \item Soutěž v IT bezpečnosti
            \item Úkoly z různých oblastí
            \item Moderní styl vzdělávání
        }
        %\icol{ctf}{CTF logo}
    \end{columns}
\end{frame}

\begin{frame}{Proč a motivace}
    \begin{columns}
        \tcol[0.7]{
            \item Naučení se pracovat s mikroslužbami
            \item Lepší pochopení Kubernetes API
            \item Pokus o on-prem řešení typicky cloudového problému
            \item Inspirace
                \begin{itemize}
                    \item CTFd
                    \item Soutěže Haxagon Skirmish a Kybersoutež
                \end{itemize}
            \item Zájem o CTF
        }
        \icol[0.3]{kubernetes}{Kubernetes logo}
    \end{columns}
\end{frame}

\begin{frame}{Požadavky projektu}
    \begin{columns}
        \tcol{
            \item Webové rozhraní
            \item Úlohy spustitelné na Kubernetes
            \item Komunikace přes webshell
            \item REST API
            \item Spustitelné v prostředí školního Kubernetes clusteru
            \item Rozšiřitelnost
            \item Stateless
        }
        \icol{deployments}{Deployments in Kubernetes}
    \end{columns}
\end{frame}

\begin{frame}{Technologie}
    \begin{columns}
        \tcol{
            \item React
            \item Kubernetes
            \item FastAPI
            \item SQLAlchemy
            \item JWT
            \item Redis
            \item Mikroslužby
        }
        %\icol{react}{React logo}
    \end{columns}
\end{frame}

\begin{frame}[fragile]{Databáze}
    \begin{columns}
        \tcol{
            \item Hlavní = PostgreSQL
            \item Pro aktuálně spuštěné úkoly = Redis
        }
        \icol{dbconnect}{Database connection}

        %\icol{redis}{Redis logo}
    \end{columns}
\end{frame}

\begin{frame}{Problémy}
    \begin{columns}
        \tcol{
            \item Logy z Access boxu
            \item Cookies
            \item Routing mezi frontendem a backendem pomocí Ingress
            \item Komunikace s Kubernetes
        }
        \icol{accessboxproblem}{Řešení problému s logy}
    \end{columns}
\end{frame}

\begin{frame}{Backend služby}
    \begin{columns}
        \tcol{
        \item Router
        \item Auth (klíč, PostgreSQL)
        \item Admin (k8s, PostgreSQL)
        \item Deployer (k8s, Redis, PostgreSQL)
        \item Lister (Redis, PostgreSQL)
        \item Deleter (k8s, Redis)
        \item Flag submit (PostgreSQL)
        }
        \icol{listerlogs}{Logy listeru}
    \end{columns}
\end{frame}

\begin{frame}{Access box}
    \begin{columns}
        \tcol{
            \item Porovnání s VPN nebo přímým přístupem
            \item Kali Linux kontejner
            \item Webový přístup
            \item Přístup k úkolům
        }
        \icol{accessbox}{Access box}
    \end{columns}
\end{frame}


\begin{frame}{Nesplněné cíle a budoucí rozšíření}
    \begin{columns}
        \tcol{
            \item Dynamické skóre
            \item Leaderboard
            \item Seznam již vyřešených úkolů
            \item Zobrazení bodů
            \item Vylepšené chybové hlášky
            \item Bezpečnostně nepředpokládat oddělený cluster pro úkoly
            \item Ověřování Kubernetes certifikátu
        }
        %\icol{requirements}{Requirements logo
    \end{columns}
\end{frame}

%\begin{frame}{Vlnově částicový dualismus}
    %\cols{
        %\item Základní pojem kvantové mechaniky
        %\item Popis chování částic na mikrofyzikální úrovni
        %\item Chování jak jako vlny, tak jako částice
        %\item Projev u světla
        %\item Popsatelné rovnicí \( \lambda = \frac{h}{p} \)
    %}{doubleslit-interference}{Interference způsobená vlnovým dualismem}
%\end{frame}

%\begin{frame}{Historie}
    %\cols{
    %\item První spory - 17. století
    %\item Youngův dvouštěrbinový experiment - 1801
    %\item Rozpor mezi vědci
    %\item Einsteinova práce u fotoelektrického jevu - 1905
    %\item De Broglieho hypotéza - 1924
    %}{
        %thomas-young
    %}{
        %Thomas Young
    %}
%\end{frame}

%\begin{frame}{Experimenty - Dvouštěrbinový}
    %\cols{
        %\item Dvouštěrbinový experiment Thomase Younga
        %\item 1801
        %\item Světlo posíláno skrz dvě štěrbiny
        %\item Interferenční obrazec
        %\item Důkaz vlnové povahy světla
    %}{
        %doubleslit
    %}{
        %Dvouštěrbinový experiment
    %}
%\end{frame}
%\begin{frame}{Experimenty - Davissonův-Germerův}
    %\cols{
    %\item Davissonův–Germerův experiment
        %\item 1927
        %\item Elektrony posílany kolmo na krystal niklu
        %\item Závislost úhlu rozptylu \(\theta\) na intenzitě rozptýleného světla
        %\item Rozptyl zaznamenán - potvrzení de Broglieho hypotézy
        %}{
        %davisson-germer
    %}{
        %Davissonův-Germerův experiment
    %}
%\end{frame}

%\begin{frame}{Využití}
    %\cols{
    %\item Fotoelektrický jev
    %\item Uvolňování elektronů z pevných látek
    %\item Solární panely
    %}{
        %photoelectric
    %}{
        %Využití
    %}
%\end{frame}

%\begin{frame}{Schrödingerova rovnice}
    %\begin{columns}
        %\tcol{
        %\item Nejdůležitější rovnice (\ref{eq:schrodinger}) kvantové mechaniky
            %\item Popis vztahu mezi vlnovou funkcí částice a její energií
            %\item Vyplývá z vlnové částicové dualismu
            %\item Poptřeba doplnit Hamiltonián  (\ref{eq:hamiltionan})
            %\item Potřeba doplnit dle prostředí
            %}
            %\col{
                %\begin{equation} \label{eq:schrodinger}
                    %i \hslash \frac {\partial} {\partial t} | \Psi (t)\rangle = \hat{H} | \Psi (t)\rangle 
                %\end{equation}
                %\begin{align*} \label{eq:hamiltionan} \numberthis
                    %\hat{H} &= - \frac{\hslash^2} {2m} \Delta + V \\
                    %\hat{H} &= - \frac{\hslash^2} {2m} \frac{\partial^2} {\partial x^2} + V \\
                %\end{align*}
            %}
    %\end{columns}

%\end{frame}

%\begin{frame}{Heisenbergův princip neurčitosti}
    %\begin{columns}
        %\tcol{
        %\item Formulován v roce 1927 Wernerem Heisenbergem
        %\item Nelze měřit hybnost a polohu částice souběžně
        %\item Podobá se dualismu - potřeba kolaps vlnové funkce - nelze měřit vlny a částice souběžně v ten samý okamžik
        %}
        %\col{
            %\begin{equation} \label{eq:heisenberg}
                %\Delta x \Delta p_x \geq \frac{\hslash}{2}
            %\end{equation}
        %}
    %\end{columns}
%\end{frame}

\begin{frame}[plain,noframenumbering]{}
    \vspace*{\fill}
    \centering \Huge
    \emph{Děkuji za pozornost}
    \vfill
\end{frame}

\end{document}
